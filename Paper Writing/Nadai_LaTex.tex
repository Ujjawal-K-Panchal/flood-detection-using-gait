% Generated by GrindEQ Word-to-LaTeX 
\documentclass{article} %%% use \documentstyle for old LaTeX compilers

\usepackage[english]{babel} %%% 'french', 'german', 'spanish', 'danish', etc.
\usepackage{amssymb}
\usepackage{amsmath}
\usepackage{txfonts}
\usepackage{mathdots}
\usepackage[classicReIm]{kpfonts}
\usepackage[dvips]{graphicx} %%% use 'pdftex' instead of 'dvips' for PDF output

% You can include more LaTeX packages here 


\begin{document}

%\selectlanguage{english} %%% remove comment delimiter ('%') and select language if required


\noindent \_\_Grammarly\_42\_\_\_\_iH4sIAAAAAAAEAKtWckksSQxILCpxzi/NK1GyMqwFAAEhoTITAAAA\_\_Grammarly\_42\_\_\_1H4sIAAAAAAAEAKtWcslP9kxRslIyNDY0tzQ1NTUwNjA1M7U0sbBU0lEKTi0uzszPAykwrAUAEX4YJCwAAAA=

\noindent 

\noindent 

\noindent 

\noindent XXX-X-XXXX-XXXX-X/XX/\$XX.00 {\copyright}20XX IEEE

\noindent \textit{522864556Nadai - Flood Analysis using smartphone sensors}522864556Hardik Ajmani522864556-2023179765changed the title\textit{}

\noindent 

\noindent line 1: 1${}^{st}$ Saad Yunus Sait line 2: \textit{Computer Science Engineering(Research Associate)}line 3: \textit{SRM Institute of Science and Technology (of Affiliation)}line 4: Chennai, Indialine 5: saad.y@ktr.srmuniv.ac.in

\noindent  2: \textit{dept. name of organization}line 1: 2${}^{nd}$ Ujjawal Kamal Panchalline 2: \textit{Computer Science Engineering (Student)}line 3: \textit{SRM Institute of Science and Technology                                          (of Affiliation)}line 4: Chennai, Indialine 5: ujjawalpanchal32@gmail.com

\noindent line 1: 5${}^{th}$ Given Name Surnameline 2: \textit{dept. name of organization (of Affiliation)}line 3: \textit{name of organization (of Affiliation)}line 4: City, Countryline 5: email address line 1: 3${}^{rd}$ Hardik Ajmaniline 2: \textit{Computer Science Engineering (Student)}line 3: \textit{SRM Institute of Science and Technology (of Affiliation)}line 4: Chennai, Indialine 5: hardikajmani@gmail.com

\noindent line 1: 6${}^{th}$ Given Name Surnameline 2: \textit{dept. name of organization (of Affiliation)}line 3: \textit{name of organization (of Affiliation)}line 4: City, Countryline 5: email address 

\textbf{\textit{Abstract}---This electronic document is a ``live'' template and already defines the components of your paper [title, text, heads, etc.] in its style sheet.  \textit{*CRITICAL:  Do Not Use Symbols, Special Characters, Footnotes, or Math in Paper Title or Abstract}. (}\textit{Abstract}\textbf{)}

\textbf{\textit{Keywords---component, formatting, style, styling, insert (}}\textit{key words}\textbf{\textit{)}}


\section{ Introduction (Heading 1)}

One practical problem faced today by many people living in places with high amount of precipitation or places which get flooded easily is the problem of water logging. Situations such as these often cause inconvenience and delays in transportation which is a major concern to government, law-enforcement, medical institutions and business corporations. Delay in transportation can also sometimes cause major problems in situations of emergency.

\noindent This paper aims to analyse the waterlogged surrounding using general sensors present in smart phones which are very commonly held and present on hand during situations of waterlogging and flood.

Using sensors such as Accelerometer, Gyroscope, Proximity Sensor, Gravitational and various other sensors present in the app AndroSensor, the height of the water in which the person with the smartphone is travelling on foot is classified using a machine learning multi class classifier. $\mathrm{<}$ Info. To be added about the model$\mathrm{>}$. The information and the conclusion drawn from the findings and research on this paper can be used to practically deal with situations of flood and water logging in urbanized environments.


\section{ Ease of Use}


\subsection{ Selecting a Template (Heading 2)}

First, confirm that you have the correct template for your paper size. This template has been tailored for output on the A4 paper size. If you are using US letter-sized paper, please close this file and download the Microsoft Word, Letter file.


\subsection{ Maintaining the Integrity of the Specifications}

The template is used to format your paper and style the text. All margins, column widths, line spaces, and text fonts are prescribed; please do not alter them. You may note peculiarities. For example, the head margin in this template measures proportionately more than is customary. This measurement and others are deliberate, using specifications that anticipate your paper as one part of the entire proceedings, and not as an independent document. Please do not revise any of the current designations.


\section{ Prepare Your Paper Before Styling}

Before you begin to format your paper, first write and save the content as a separate text file. Complete all content and organizational editing before formatting. Please note sections A-D below for more information on proofreading, spelling and grammar.

Keep your text and graphic files separate until after the text has been formatted and styled. Do not use hard tabs, and limit use of hard returns to only one return at the end of a paragraph. Do not add any kind of pagination anywhere in the paper. Do not number text heads-the template will do that for you.


\subsection{ Abbreviations and Acronyms}

Define abbreviations and acronyms the first time they are used in the text, even after they have been defined in the abstract. Abbreviations such as IEEE, SI, MKS, CGS, sc, dc, and rms do not have to be defined. Do not use abbreviations in the title or heads unless they are unavoidable.


\subsection{ Units}

\begin{enumerate}
\item \textit{ }Use either SI (MKS) or CGS as primary units. (SI units are encouraged.) English units may be used as secondary units (in parentheses). An exception would be the use of English units as identifiers in trade, such as ``3.5-inch disk drive''.

\item  Avoid combining SI and CGS units, such as current in amperes and magnetic field in oersteds. This often leads to confusion because equations do not balance dimensionally. If you must use mixed units, clearly state the units for each quantity that you use in an equation.

\item  Do not mix complete spellings and abbreviations of units: ``Wb/m2'' or ``webers per square meter'', not ``webers/m2''.  Spell out units when they appear in text: ``. . . a few henries'', not ``. . . a few H''.
\end{enumerate}

Identify applicable funding agency here. If none, delete this text box.

\begin{enumerate}
\item  Use a zero before decimal points: ``0.25'', not ``.25''. Use ``cm3'', not ``cc''. (\textit{bullet list})
\end{enumerate}


\subsection{ Equations}

The equations are an exception to the prescribed specifications of this template. You will need to determine whether or not your equation should be typed using either the Times New Roman or the Symbol font (please no other font). To create multileveled equations, it may be necessary to treat the equation as a graphic and insert it into the text after your paper is styled.

Number equations consecutively. Equation numbers, within parentheses, are to position flush right, as in \eqref{GrindEQ__1_}, using a right tab stop. To make your equations more compact, you may use the solidus ( / ), the exp function, or appropriate exponents. Italicize Roman symbols for quantities and variables, but not Greek symbols. Use a long dash rather than a hyphen for a minus sign. Punctuate equations with commas or periods when they are part of a sentence, as in:

 \textit{a}???\textit{b}????? ???

Note that the equation is centered using a center tab stop. Be sure that the symbols in your equation have been defined before or immediately following the equation. Use ``\eqref{GrindEQ__1_}'', not ``Eq. \eqref{GrindEQ__1_}'' or ``equation \eqref{GrindEQ__1_}'', except at the beginning of a sentence: ``Equation \eqref{GrindEQ__1_} is . . .''


\subsection{ Some Common Mistakes}

\begin{enumerate}
\item \textit{ }The word ``data'' is plural, not singular.

\item  The subscript for the permeability of vacuum \textit{$\mu$}${}_{0}$, and other common scientific constants, is zero with subscript formatting, not a lowercase letter ``o''.

\item  In American English, commas, semicolons, periods, question and exclamation marks are located within quotation marks only when a complete thought or name is cited, such as a title or full quotation. When quotation marks are used, instead of a bold or italic typeface, to highlight a word or phrase, punctuation should appear outside of the quotation marks. A parenthetical phrase or statement at the end of a sentence is punctuated outside of the closing parenthesis (like this). (A parenthetical sentence is punctuated within the parentheses.)

\item  A graph within a graph is an ``inset'', not an ``insert''. The word alternatively is preferred to the word ``alternately'' (unless you really mean something that alternates).

\item  Do not use the word ``essentially'' to mean ``approximately'' or ``effectively''.

\item  In your paper title, if the words ``that uses'' can accurately replace the word ``using'', capitalize the ``u''; if not, keep using lower-cased.

\item  Be aware of the different meanings of the homophones ``affect'' and ``effect'', ``complement'' and ``compliment'', ``discreet'' and ``discrete'', ``principal'' and ``principle''.

\item  Do not confuse ``imply'' and ``infer''.

\item  The prefix ``non'' is not a word; it should be joined to the word it modifies, usually without a hyphen.

\item  There is no period after the ``et'' in the Latin abbreviation ``et al.''.

\item  The abbreviation ``i.e.'' means ``that is'', and the abbreviation ``e.g.'' means ``for example''.
\end{enumerate}

An excellent style manual for science writers is [7].


\section{ Using the Template}

After the text edit has been completed, the paper is ready for the template. Duplicate the template file by using the Save As command, and use the naming convention prescribed by your conference for the name of your paper. In this newly created file, highlight all of the contents and import your prepared text file. You are now ready to style your paper; use the scroll down window on the left of the MS Word Formatting toolbar.


\subsection{ Authors and Affiliations}

\textbf{The template is designed for, but not limited to, six authors.} A minimum of one author is required for all conference articles. Author names should be listed starting from left to right and then moving down to the next line. This is the author sequence that will be used in future citations and by indexing services. Names should not be listed in columns nor group by affiliation. Please keep your affiliations as succinct as possible (for example, do not differentiate among departments of the same organization).


\paragraph{ For papers with more than six authors: Add author names horizontally, moving to a third row if needed for more than 8 authors.}


\paragraph{ For papers with less than six authors: To change the default, adjust the template as follows.}


\subparagraph{ Selection: Highlight all author and affiliation lines.}


\subparagraph{ Change number of columns: Select the Columns icon from the MS Word Standard toolbar and then select the correct number of columns from the selection palette.}


\subparagraph{ Deletion: Delete the author and affiliation lines for the extra authors.}

\noindent 


\subsection{ Identify the Headings}

Headings, or heads, are organizational devices that guide the reader through your paper. There are two types: component heads and text heads.

Component heads identify the different components of your paper and are not topically subordinate to each other. Examples include Acknowledgments and References and, for these, the correct style to use is ``Heading 5''. Use ``figure caption'' for your Figure captions, and ``table head'' for your table title. Run-in heads, such as ``Abstract'', will require you to apply a style (in this case, italic) in addition to the style provided by the drop down menu to differentiate the head from the text.

Text heads organize the topics on a relational, hierarchical basis. For example, the paper title is the primary text head because all subsequent material relates and elaborates on this one topic. If there are two or more sub-topics, the next level head (uppercase Roman numerals) should be used and, conversely, if there are not at least two sub-topics, then no subheads should be introduced. Styles named ``Heading 1'', ``Heading 2'', ``Heading 3'', and ``Heading 4'' are prescribed.


\subsection{ Figures and Tables}


\subparagraph{  Positioning Figures and Tables: Place figures and tables at the top and bottom of columns. Avoid placing them in the middle of columns. Large figures and tables may span across both columns. Figure captions should be below the figures; table heads should appear above the tables. Insert figures and tables after they are cited in the text. Use the abbreviation ``Fig. 1'', even at the beginning of a sentence.}

\begin{enumerate}
\item \textit{ }Table Type Styles
\end{enumerate}

\begin{tabular}{|p{0.3in}|p{1.1in}|p{0.4in}|p{0.4in}|} \hline 
\textbf{Table Head} & \multicolumn{3}{|p{1.9in}|}{\textbf{Table Column Head}} \\ \hline 
\textbf{} & \textbf{\textit{Table column subhead}} & \textbf{\textit{Subhead}} & \textbf{\textit{Subhead}} \\ \hline 
copy & More table copy${}^{a}$ &  &  \\ \hline 
\end{tabular}

\begin{enumerate}
\item  Sample of a Table footnote. (\textit{Table footnote})

\item  Example of a figure caption. (\textit{figure caption})
\end{enumerate}

Figure Labels: Use 8 point Times New Roman for Figure labels. Use words rather than symbols or abbreviations when writing Figure axis labels to avoid confusing the reader. As an example, write the quantity ``Magnetization'', or ``Magnetization, M'', not just ``M''. If including units in the label, present them within parentheses. Do not label axes only with units. In the example, write ``Magnetization (A/m)'' or ``Magnetization $\mathrm{\{}$A[m\eqref{GrindEQ__1_}]$\mathrm{\}}$'', not just ``A/m''. Do not label axes with a ratio of quantities and units. For example, write ``Temperature (K)'', not ``Temperature/K''.

\noindent 
{\bf Acknowledgment (Heading 5)}

The preferred spelling of the word ``acknowledgment'' in America is without an ``e'' after the ``g''. Avoid the stilted expression ``one of us (R. B. G.) thanks ...''.  Instead, try ``R. B. G. thanks...''. Put sponsor acknowledgments in the unnumbered footnote on the first page.

\noindent 
{\bf References}

The template will number citations consecutively within brackets [1]. The sentence punctuation follows the bracket [2]. Refer simply to the reference number, as in [3]---do not use ``Ref. [3]'' or ``reference [3]'' except at the beginning of a sentence: ``Reference [3] was the first ...''

Number footnotes separately in superscripts. Place the actual footnote at the bottom of the column in which it was cited. Do not put footnotes in the abstract or reference list. Use letters for table footnotes.

Unless there are six authors or more give all authors' names; do not use ``et al.''. Papers that have not been published, even if they have been submitted for publication, should be cited as ``unpublished'' [4]. Papers that have been accepted for publication should be cited as ``in press'' [5]. Capitalize only the first word in a paper title, except for proper nouns and element symbols.

For papers published in translation journals, please give the English citation first, followed by the original foreign-language citation [6].

\noindent 

\begin{enumerate}
\item  G. Eason, B. Noble, and I. N. Sneddon, ``On certain integrals of Lipschitz-Hankel type involving products of Bessel functions,'' Phil. Trans. Roy. Soc. London, vol. A247, pp. 529--551, April 1955. \textit{(references)}

\item  J. Clerk Maxwell, A Treatise on Electricity and Magnetism, 3rd ed., vol. 2. Oxford: Clarendon, 1892, pp.68--73.

\item  I. S. Jacobs and C. P. Bean, ``Fine particles, thin films and exchange anisotropy,'' in Magnetism, vol. III, G. T. Rado and H. Suhl, Eds. New York: Academic, 1963, pp. 271--350.

\item  K. Elissa, ``Title of paper if known,'' unpublished.

\item  R. Nicole, ``Title of paper with only first word capitalized,'' J. Name Stand. Abbrev., in press.

\item  Y. Yorozu, M. Hirano, K. Oka, and Y. Tagawa, ``Electron spectroscopy studies on magneto-optical media and plastic substrate interface,'' IEEE Transl. J. Magn. Japan, vol. 2, pp. 740--741, August 1987 [Digests 9th Annual Conf. Magnetics Japan, p. 301, 1982].

\item  M. Young, The Technical Writer's Handbook. Mill Valley, CA: University Science, 1989.

\item  
\end{enumerate}

 \textbf{IEEE conference templates contain guidance text for composing and formatting conference papers. Please ensure that all template text is removed from your conference paper prior to submission to the conference. Failure to remove template text from your paper may result in your paper not being published.}


\end{document}

